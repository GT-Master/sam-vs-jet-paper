%% Version 4.3.1, 19 May 2014
%
%%%%%%%%%%%%%%%%%%%%%%%%%%%%%%%%%%%%%%%%%%%%%%%%%%%%%%%%%%%%%%%%%%%%%%
% Template.tex --  LaTeX-based template for submissions to the 
% American Meteorological Society
%
% Template developed by Amy Hendrickson, 2013, TeXnology Inc., 
% amyh@texnology.com, http://www.texnology.com
% following earlier work by Brian Papa, American Meteorological Society
%
% Email questions to latex@ametsoc.org.
%
%%%%%%%%%%%%%%%%%%%%%%%%%%%%%%%%%%%%%%%%%%%%%%%%%%%%%%%%%%%%%%%%%%%%%
% PREAMBLE
%%%%%%%%%%%%%%%%%%%%%%%%%%%%%%%%%%%%%%%%%%%%%%%%%%%%%%%%%%%%%%%%%%%%%

%% Start with one of the following:
% DOUBLE-SPACED VERSION FOR SUBMISSION TO THE AMS
\documentclass{ametsoc}

% TWO-COLUMN JOURNAL PAGE LAYOUT---FOR AUTHOR USE ONLY
% \documentclass[twocol]{ametsoc}

%%%%%%%%%%%%%%%%%%%%%%%%%%%%%%%%
%%% To be entered only if twocol option is used

\journal{jcli}

%  Please choose a journal abbreviation to use above from the following list:
% 
%   jamc     (Journal of Applied Meteorology and Climatology)
%   jtech     (Journal of Atmospheric and Oceanic Technology)
%   jhm      (Journal of Hydrometeorology)
%   jpo     (Journal of Physical Oceanography)
%   jcli      (Journal of Climate)
%   mwr      (Monthly Weather Review)
%   wcas      (Weather, Climate, and Society)
%   waf       (Weather and Forecasting)
%   bams (Bulletin of the American Meteorological Society)
%   ei    (Earth Interactions)

%%%%%%%%%%%%%%%%%%%%%%%%%%%%%%%%
%Citations should be of the form ``author year''  not ``author, year''
\bibpunct{(}{)}{;}{a}{}{,}

%%%%%%%%%%%%%%%%%%%%%%%%%%%%%%%%

%%% To be entered by author:

%% May use \\ to break lines in title:

\title{Comparing trends in the Southern Annular Mode and surface westerly jet}

%%% Enter authors' names, as you see in this example:
%%% Use \correspondingauthor{} and \thanks{Current Affiliation:...}
%%% immediately following the appropriate author.
%%%
%%% Note that the \correspondingauthor{} command is NECESSARY.
%%% The \thanks{} commands are OPTIONAL.

    %\authors{Author One\correspondingauthor{Author One, 
    % American Meteorological Society, 
    % 45 Beacon St., Boston, MA 02108.}
% and Author Two\thanks{Current affiliation: American Meteorological Society, 
    % 45 Beacon St., Boston, MA 02108.}}

\authors{Neil C. Swart\correspondingauthor{Canadian Center for Climate Modelling and Analysis,
 Environment Canada, University of Victoria, P.O. Box 1700 STN CSC, Victoria, BC, V8W 2Y2, Canada}, 
 John C. Fyfe and Nathan Gillett}

%% Follow this form:
    % \affiliation{American Meteorological Society, 
    % Boston, Massachusetts.}

\affiliation{Canadian Center for Climate Modelling and Analysis, Victoria, BC, Canada}

%% Follow this form:
    %\email{latex@ametsoc.org}

\email{Neil.Swart@ec.gc.ca}

%% If appropriate, add additional authors, different affiliations:
    \extraauthor{Gareth J. Marshall}
    \extraaffil{British Antarctic Survey, Cambridge, Cambridgeshire, United Kingdom}

%\extraauthor{}
%\extraaffil{}

%% May repeat for a additional authors/affiliations:

%\extraauthor{}
%\extraaffil{}



%%%%%%%%%%%%%%%%%%%%%%%%%%%%%%%%%%%%%%%%%%%%%%%%%%%%%%%%%%%%%%%%%%%%%
% ABSTRACT
%
% Enter your Abstract here

\abstract{We examine trends in the Southern Annular Mode (SAM), and the strength, position and width 
of the Southern Hemisphere surface westerly wind jet in observations, reanalyses and 
the Coupled Model Intercomparison Phase 5 (CMIP5) models. First we consider the period over 1951 to 2011, 
and show that there are differences in the SAM and jet trends between the CMIP5 models,
the HadSLP2r gridded sea-level pressure (SLP) dataset, and the Twentieth Century Reanalysis. The relationships between
these trends demonstrates that the SAM index cannot be used to directly infer changes in any one kinematic 
property of the jet. The spatial structure of the observed trends in SLP and zonal winds is shown to be largest, but also
most uncertain, in the southeast Pacific. To constrain this uncertainty we include six reanalyses 
and do comparisons with station based observations of SLP. We find the CMIP5 mean SLP trends generally agree well with the 
direct observations, despite some climatological 
biases, while some reanalyses exhibit spuriously large SLP trends. Similarly, over the more reliable satellite era 
the spatial pattern of CMIP5 SLP trends is in excellent agreement with HadSLP2r, while several reanalyses
are not. Then we compare surface winds with a satellite based product, and show that the CMIP5 mean trend is 
similar to observed in the core region of the westerlies, but that several reanalyses overestimate recent trends.  
We caution that studies examining the impact of wind changes on the Southern ocean could 
be biased by these spuriously large trends in reanalysis products.}
\begin{document}

%% Necessary!
\maketitle

\section{Introduction}
The Southern Hemisphere (SH) westerlies are the strongest time-averaged surface winds on the planet,  and they 
exert a pronounced influence on the global climate system. They do so in part by driving upwelling of deep waters
in the Southern Ocean, and thereby the upper limb of the Atlantic Meridional Overturning Circulation (AMOC)
\citep{Toggweiler_Samuels_1995, Marshall_and_Speer_2012}. The AMOC, in turn, strongly modulates the oceanic 
uptake of heat and carbon \citep{Kostov_et_al_2014, Frolicher_et_al_2014}, as well as controlling global 
primary production through regulation of the nutrient supply to the ocean thermocline \citep{Sarmiento_et_al_2003, 
Marinov_et_al_2006}. Variability and changes in the westerlies are thus of central interest when considering 
human induced climate change \citep{Toggweiler_and_Russell_2008}.

The dominant mode of atmospheric variability in the SH is the Southern Annular Mode (SAM). The SAM index
has alternately been characterized as the leading Empirical Orthogonal Function (EOF) of sea-level pressure in the SH
\citep{Thompson_and_Wallace_2000} and as the sea-level pressure
difference between 40$^{\circ}$ and 65$^{\circ}$S \citep{Gong_and_Wang_1999}. Observations have shown a trend
towards the positive phase of the SAM since about 1970 \citep{Thompson_and_Solomon_2002, Marshall_2003}. Modelling
studies have attributed this trend to human influence from a combination of increasing greenhouse gases
and ozone depletion \citep{Fyfe_et_al_1999, Son_et_al_2010, Gillett_et_al_2013}. The influence of ozone depletion
has a strong seasonal signal, being largest during the Austral summer (December-January-February or DJF), 
whereas the greenhouse gas forcing operates consistently year round 
\citep{Son_et_al_2010, Thompson_et_al_2011, Gillett_et_al_2013}. As a result, historical trends in the SAM
are largest during the Austral summer, but small and 
statistically insignificant during the Austral winter \citep{Thompson_et_al_2011}.

These recent trends in the SAM have been associated with changes in the tropospheric circulation and climate 
\citep{Thompson_and_Solomon_2002, Thompson_et_al_2011}. Month-to-month changes in the polarity of the SAM index are
primarily associated with nearly symmetrical north-south vacillations of the surface westerly jet (herein referred
to simply as the jet)
\citep{Hartmann_and_Lo_1998, Thompson_and_Wallace_2000}. The positive phase of the SAM is associated
with a poleward shifted jet, such that the westerlies are stronger over much of the Southern Ocean 
(with a centre near 60$^{\circ}$S) and weaker to the north (with a centre near 40$^{\circ}$S) 
\citep{Thompson_et_al_2011}. However, oscillations in the SAM are also associated with changes 
in the width of the westerly jet and the strength of the jet at its peak \citep{Monahan_and_Fyfe_2006}. 
Indeed, the historical trend towards the positive phase of the SAM during the Austral summer has 
been concurrent with both a poleward shift and a strengthening at the peak of the westerly 
jet \citep{Swart_and_Fyfe_2012b}. 

The climate models participating in the Coupled Modelling Intercomparison Project (CMIP) phase 3 and phase 5
show systematic biases in their simulation of the SH westerly jet. On average the models simulate a
climatological jet position that is 2$^{\circ}$ to 3$^{\circ}$ of latitude equatorward of the observed 
position over the historical period \citep{Swart_and_Fyfe_2012b, Bracegirdle_et_al_2013}. 
\cite{Swart_and_Fyfe_2012b} also showed that the simulated
trends in jet strength  over 1979 to 2010 were significantly smaller at the 5\% level 
than the trends seen in the average of four reanalysis products (R1, R2, 20CR and ERA-Int; see 
Table \ref{t:rean_list}) 
in all seasons except JJA. However, they also cautioned that this result was potentially unreliable, given 
that the reanalyses showed a large spread of trends and were poorly constrained in the Southern Hemisphere 
\citep{Swart_and_Fyfe_2012b}.

More recently \cite{Gillett_and_Fyfe_2013} showed that over 1951 to 2011 the CMIP5 models simulate a SAM trend 
which was consistent with observationally based estimates, at least during DJF. Since 
trends in the strength of the westerly jet should be closely related to those in the SAM index (or sea-level 
pressure gradient) through geostrophy, the findings of \citet{Swart_and_Fyfe_2012b} and \cite{Gillett_and_Fyfe_2013}
 appear to be contradictory. However, given
that the studies covered different time-frames and used different metrics, there are many potential 
reasons for the apparent contradiction. In this paper we will compare changes in both the SAM and the 
westerly jet over a common period to resolve this discrepancy.

The aims of this study are to address two principal questions: 1) what is the relationship  between trends 
in the SAM index and the kinematic properties of the westerly jet? and 2) how do
historical trends in the SAM and westerly jet compare between the best available direct observations,
common reanalysis products and the CMIP5 climate models? The second question is designed to quantify 
any systematic biases in the reanalyses or CMIP5 models. A major difficulty is that the direct observational
estimates of sea-level pressure and winds are not available with comprehensive coverage in both space and time. 
Here we attempt to make the closest possible comparison with the best available observations, which requires
comparing trends in the SAM and winds over several different periods, and at specific geographic locations.

In the following section we describe the data and methods used in this study. Section \ref{sec:obs_sim_changes} 
begins  by considering changes in the SAM index and kinematic properties of the westerly jet 
focusing on the historical period since 1951. We start with a long historical record (i.e. pre-satellite),
because it facilitates the robust detection of long-term trends and it also allows us to compare our results with 
\cite{Gillett_and_Fyfe_2013}. Section \ref{sec:sam_vs_jet} uses a simple theoretical model to 
establish the expected relationship between SAM changes and jet properties and  shows that this simple 
description largely explains the relationships seen in the full CMIP5 models. The spatial pattern of trends
is examined in Section \ref{sec:spatial_patterns}. Then, in Section \ref{sec:obs-intercomp} we
undertake a detailed intercomparison of changes in sea-level pressure and surface winds in various observations,
reanalysis products and the CMIP5 models over the more recent and reliable satellite era. In the final section
we synthesise our findings and draw some broader conclusions.

\section{Data and methods} \label{sec:data_and_methods}
We use sea-level pressure, 10 meter zonal wind speed fields (u10m) and surface eastward wind stress
from ensemble member 1 from 30 CMIP5 models 
(ACCESS1-0, ACCESS1-3, bcc-csm1-1, bcc-csm1-1-m, BNU-ESM,
CanESM2, CMCC-CM, CMCC-CMS, CNRM-CM5, CSIRO-Mk3-6-0, GISS-E2-H, GISS-E2-H-CC, GISS-E2-R, 
GISS-E2-R-CC, HadCM3, HadGEM2-AO, HadGEM2-CC, HadGEM2-ES, inmcm4, IPSL-CM5A-LR, IPSL-CM5A-MR,
IPSL-CM5B-LR, MIROC5, MIROC-ESM, MIROC-ESM-CHEM, MPI-ESM-LR, MPI-ESM-MR, MRI-CGCM3, 
NorESM1-M and NorESM1-ME). We also use the equivalent output from 6 reanalyses, listed with 
their abbreviations, references and the data source in Table \ref{t:rean_list}. The 
Twentieth Century Reanalysis \citep[20CR;][]{Compo_et_al_2011}, is an ensemble reanalysis
consisting of 56 members. For both CMIP5 and 20CR we perform our analysis on the 
individual ensemble members, and then compute an ensemble mean with an associated uncertainty
(see below).

We use the gridded observational sea-level pressure dataset, HadSLP2r, 
with reduced variance \citep{Allan_and_Ansell_2006}. HadSLP2 extends from 1850 to 2004 
and is based on quality controlled marine and terrestrial pressure observations that
have been blended, gridded, and made spatially complete using a reduced space optimal
interpolation. HadSLP2r extends this from 2005 to 2012 based 
on NCEP-NCAR R1 fields (Table \ref{t:rean_list}), which have been adjusted 
to have the same mean and variance as HadSLP2. This ``reduced variance''
version is available online at \url{http://www.metoffice.gov.uk/hadobs/hadslp2}. We also 
use the observed sea-level pressures over 1958 to 2011 
updated from \cite{Marshall_2003}. \cite{Marshall_2003} used 12 individual stations 
to compute the proxy zonal mean SLPs at 40$^{\circ}$S and 65$^{\circ}$S (six stations
near each latitude circle). Additional observationally-based SAM reconstructions exist
\citep[e.g.][]{Jones_et_al_2009, Visbeck_2009}, and have previously been compared with each other
\citep{Ho_et_al_2012}, but we do not make use of them here.

The Cross-Calibrated Multi-Platform (CCMP) Ocean Surface Wind Vector Analyses of 
\cite{Atlas_et_al_2011} is used for u10m winds and psuedo-windstress fields over the period
1988 to 2011. The data were downloaded from the Research Data Archive at the National 
Center for Atmospheric Research, Computational and Information Systems Laboratory, Boulder, CO. 
[Available online at \url{http://rda.ucar.edu/datasets/ds744.9/}]. 
The supplied zonal psuedo-windstress ($u^2$) is converted to wind-stress 
as: $\tau_x = \rho\; c_d\; u^2$,
where $\rho=1.2$ kg m$^{-3}$ is the density of air and $c_d=1.4\times10^{-3}$ is a dimensionless
drag coefficient. CCMP is created using a variational analysis method (VAM), which takes in data
from satellite radiometers and scatterometers, as well as ship and buoy observations. Observations
are adjusted to the 10m level assuming neutral stability. The VAM combines the data in a best fit, 
while satisfying smoothness and dynamical constraints. The procedure also requires a first-guess 
field, which comes from the ERA-40 reanalysis from July 1987 to December 1998, and from ERA-Interim 
thereafter \citep{Atlas_et_al_2011}. Here we refer to CCMP as ``satellite observations``, while
acknowledging the presence of other observational inputs, and the reanalysis-based first guess.

The non-normalized SAM index is calculated as the zonal mean sea-level pressure difference between 
40$^{\circ}$ and 65$^{\circ}$S in hPa (across all longitudes), 
as in \cite{Gillett_and_Fyfe_2013}, except where noted. Alternatively, in Figures 
\ref{fig:marshall_timeseries}, \ref{fig:marshall_trends} and where noted, the 
SAM index is calculated in the same way but using only data from the 12 locations coincident with the 
stations used by \cite{Marshall_2003}. The strength of the westerly jet is taken as the maximum of 
the zonal mean u10m between 20$^{\circ}$ and 70$^{\circ}$S in m~s$^{-1}$. The position of
the jet is taken as the latitude, in degrees, at the jet maximum. The jet width is taken as the range
of contiguous latitudes between 20$^{\circ}$ and 70$^{\circ}$S, in degrees, where the zonal mean u10m 
is positive.

Trends in the SAM index and jet properties were 
computed over various different time-intervals, to allow for comparison with different observational
products that each cover a limited period. For the CMIP5 and 20CR ensembles we also compute 
the ensemble mean and a 95\% confidence interval based on the standard deviation 
of the trends across the individual ensemble members. The 2.5$^\textrm{th}$ to 
97.5$^\textrm{th}$ percentile of trends across the individual 
members of the CMIP5 and 20CR ensembles are also given where appropriate.

The analysis carried out in this paper was performed with the aid of 
IPython \citep{Perez_and_Granger_2007}, and graphics were produced 
with matplotlib \citep{Hunter_2007} version 
1.4.3 [\url{http://dx.doi.org/10.5281/zenodo.15423}]. The analysis is
fully reproducible with the open source code 
available from the authors upon request.


\section{Observed and simulated changes in the SAM and westerly jet} \label{sec:obs_sim_changes}
\subsection{Time-series}
Over 1871 to 1950 the annual mean SAM index from 20CR, HadSLP2r, and the 
CMIP5 models hover around 25 hPa on average (Fig. \ref{fig:timeseries}a). 
Over this period, the CMIP5 ensemble mean has an equatorward biased jet position  
relative to 20CR (Fig. \ref{fig:timeseries}c), but the simulated 
jet strength and width are roughly
equivalent to those in 20CR (Fig. \ref{fig:timeseries}b, d). 

Prior to 1950, these metrics show pronounced
interannual and decadal time scale variability, but no significant secular  
trends. From around 1950 onwards, HadSLP2r and 20CR both show a clear shift towards 
larger values of the SAM index. Jet strength shows a simultaneous increase 
in 20CR over this period, while consistent changes in jet position and width are less evident. The 
CMIP5 models also show an increase in the SAM index and jet strength, although the simulated 
increase generally appears lower than that seen in the 20CR and HadSLP2. 

To more closely compare changes between 20CR, HadSLP2r and the CMIP5 models, we next consider linear 
trends in these metrics over 1951 to 2011. The NCEP-NCAR Reanalysis 1 (R1) is also available
over this period, but we exclude it here because it is known to exhibit spurious 
trends in the SAM \citep{Marshall_2003}. However, in section \ref{sec:obs-intercomp} 
we will conduct a more thorough inter-observational product comparison. 

\subsection{Linear trends by season over 1951 to 2011}
Over 1951 to 2011 both HadSLP2r and 20CR shows a positive SAM trend during all seasons 
(Fig. \ref{fig:seas_trends_1951-2011}a). The HadSLP2r SAM trends are generally a little 
smaller than those in 20CR, and exhibit more seasonality. The CMIP5 models also exhibit positive trends 
on average during all seasons, but the model trends show the opposite seasonality to HadSLP2r,
being largest in DJF and smallest in JJA on average, as would be expected from the ozone
related forcing \citep{Son_et_al_2010, Thompson_et_al_2011}. 

During DJF the model-mean SAM trend is almost identical to that seen in 20CR, consistent 
with \cite{Gillett_and_Fyfe_2013}. However, during the Austral winter (JJA) and in the annual 
mean, the models significantly underestimate the SAM trend relative to 20CR and HadSLP2r.
Significance in this sense is determined from the fact that the 20CR ensemble mean trend 
lies outside of the  2.5$^\textrm{th}$ to 97.5$^\textrm{th}$ percentile of CMIP5 trends, 
and thus we can reject the null
hypothesis that the 20CR and CMIP5 trends come from the same distribution, at the 5\% level.
Similarly, we can reject the null hypothesis that the CMIP5 and 20CR SAM trends are equal
at the 5\% level during all seasons except DJF using a Welch's t-test, which takes uncertainty 
in both the CMIP5 and 20CR trends into account.

In jet strength, 20CR exhibits a trend of between 0.15 and 0.25~m~s$^{-1}$~dec$^{-1}$ 
(Fig. \ref{fig:seas_trends_1951-2011}b). The CMIP5 models also show positive jet strength 
trends in all seasons on average. Yet for jet strength, the modelled trends are 
significantly smaller than for 20CR in all seasons, with annual mean trend being about 
5 times weaker in the models. In all seasons, the 20CR trends lie outside 
the 2.5$^\textrm{th}$ to 97.5$^\textrm{th}$ percentile of CMIP5 trends.

Trends in jet position vary in sign over the seasons in 20CR (Fig. \ref{fig:seas_trends_1951-2011}c),
with a small, non-significant trend in the annual mean. The CMIP5 models show poleward trends
in jet position that are significant at the 5\% level during all season except JJA. The largest poleward
trend in jet position occurs in DJF, with nearly identical trends in 20CR and the CMIP5 mean. Jet width
does not exhibit any significant trends in the CMIP5 models, except for in DJF which has a broadening
trend of about 0.1$^{\circ}$ latitude per decade on average. 20CR, by contrast, shows narrowing trends in
all seasons, especially SON.

The disagreements between 20CR, HadSLP2r and the CMIP5
models identified here at least partly reflect spuriously large trends in 20CR and HadSLP2r, rather than 
an underestimation of the ``true'' trend by the CMIP5 models, as we shall see in 
Sections  \ref{sec:spatial_patterns} and \ref{sec:obs-intercomp}. Regardless, our key focus 
here is to highlight that over 1951 to 2011 the DJF jet strength trends differ by more than
 a factor of two between 20CR and 
CMIP5, while their SAM trends are similar. Indeed, it is not valid to assume that trends in the 
SAM index and jet properties are directly interchangeable, as we show in the following section.

\section{The relationship between changes in the SAM index and westerly jet properties} \label{sec:sam_vs_jet}
\subsection{A simple theoretical model}
To illustrate the relationship between the SAM index and the kinematic properties of the jet,
we use a simple geostrophic model. The zonal mean zonal velocity, $U$, is 
given by a Gaussian, with a specified position, $\Phi$, strength, $\eta$ and width, $\sigma$:
\begin{equation}
U (\phi) = \eta \cdot \textrm{exp} \left( - \frac{ (\phi -\Phi)^2}{2 \sigma^2} \right )
\end{equation}
where $\phi$ is latitude. In this model, the zonal jet velocity is related to the surface pressure field 
via geostrophy, such that:
\begin{equation}
P(\phi) = - \rho \int f U  \;\delta y
\end{equation}
where $f=2\omega \,\textrm{sin} ( \phi)$ is the Coriolis parameter, given the angular rotation
rate of Earth, $\omega=7.3 \times10^{−5}$~s$^{-1}$, and $\rho=1.2$~kg~m$^{-3}$ is the 
density of air. We can use this idealized model to examine how the SAM changes are related 
to changes in an individual kinematic property of the jet, while the other kinematic 
properties are held fixed (Fig. \ref{fig:gaussian_jet}).

Changes in jet strength and the SAM index are linearly related, such that an increasing SAM is
associated with a strengthening jet (Fig. \ref{fig:gaussian_jet}d). Changes in jet position and 
the SAM index are inversely related, with a poleward shifting jet corresponding 
to a strengthening SAM (Fig. \ref{fig:gaussian_jet}e).  However, the relationship
is not identically linear. The increase in SAM is largest per unit of poleward shift for jets 
which are more equatorward. For example, for a jet that is centred at 45$^{\circ}$S a poleward shift of one 
degree latitude is associated with an increase in the SAM index of about 1.7 hPa, while 
for a jet that is centred at 50$^{\circ}$S the increase in SAM is less than 1 hPa for 
the same one degree poleward shift. Changes in the SAM index are also proportional to changes 
in jet width, but are generally more sensitive to jet narrowing than to jet widening.

The chief value of the model used here is to illustrate that changes in the SAM index can be
influenced by changes in all three kinematic properties of the jet, as found previously 
\citep{Monahan_and_Fyfe_2006, Monahan_and_Fyfe_2008}. Changes in the SAM may be associated 
with changes in one kinematic property of the jet, while the other kinematic properties 
remain constant or even change in the opposite sense. As such, it is generally not appropriate to assume that changes 
in the SAM are directly interchangeable with changes in any one 
property of the jet \citep{Monahan_and_Fyfe_2006, Monahan_and_Fyfe_2008}. 
The CMIP5 models allow us to explore these relationships further.

\subsection{SAM-jet relationships in the CMIP5 models and 20CR}
We first consider the relationships between the trends in the SAM index and the kinematic
properties of the jet for a single season, DJF, when the changes are largest. Trends 
in the DJF SAM index over 1951 to 2011 are significantly correlated with 
trends in all three kinematic properties of the jet across 
the CMIP5 models (Fig. \ref{fig:sam_vs_jet_scatter}a, c, e). The nature of the relationships 
are as predicted by the simple geostrophic model. 

The SAM index trend is also significantly correlated with the climatological 
position and inversely correlated with the climatological jet strength across the CMIP5 models 
(Fig. \ref{fig:sam_vs_jet_scatter}b, d). The correlation between
SAM index trend and climatological position was also predicted by the simple geostrophic model: 
the change in SAM index is more sensitive to changes in jet position for jets which are 
equatorward displaced. In addition, it is known that jets with a more equatorward 
climatological position experience larger historical poleward trends in position 
\citep{Kidston_and_Gerber_2010, Bracegirdle_et_al_2013}. Why stronger climatological jets 
lead to larger trends in the SAM index is unclear, and not anticipated from the simple model. 
It could result simply because climatological jet position and strength are correlated 
($r=-0.5, p=0.004$), and the SAM index is responding to the initial jet position.

Given these correlations showing that models with large SAM trends tend to have large trends 
in jet strength, position and width, it might appear that the SAM index can be used to infer
changes in the jet. However, the relationships between trends in the SAM and the kinematic
properties of the jet change by season. This is demonstrated for the relationship between
trends in the SAM index and jet strength (Fig. \ref{fig:sam_vs_jet_seas}). Further, the relationships between
the SAM and the jet differ between the CMIP5 and 20CR ensembles (Fig. \ref{fig:sam_vs_jet_scatter},
\ref{fig:sam_vs_jet_seas}), and also differ when comparing the six reanalyses in Table \ref{t:rean_list} 
to the CMIP5 models over the satellite era (not shown). The correlations between the SAM and jet 
properties within a given model also vary significantly over the CMIP5 ensemble. For example, the correlation 
between the SAM index and jet strength varies from $r=0.44$ in IPSL-CM5B-LR to $r=0.84$ in ACCESS1-0. 
Therefore, given the variability of these SAM jet relations across models and by season, trends in the SAM 
index cannot be used as a direct proxy for trends in the jet.

The reasoning above also explains how it is that 20CR-mean and the CMIP5-mean SAM 
trends can be similar,
while the 20CR mean jet strength trend is much larger than seen in the models on average 
(Fig. \ref{fig:sam_vs_jet_scatter}a and \ref{fig:seas_trends_1951-2011}). The poleward trend 
in jet position is similar between 20CR and the models
on average (Fig. \ref{fig:sam_vs_jet_scatter}c), however, the models show a positive jet width trend 
(broadening) on average, while 20CR shows a small negative width trend on average 
(Fig. \ref{fig:sam_vs_jet_scatter}e). Thus,
the broadening of the jet in the models makes it dynamically consistent for them to have the same 
SAM trend as 20CR, even though their jet strength trends are much weaker than in 20CR. In addition, 
the models have a equatorward biased climatological jet position relative to 20CR, and more 
equatorward jets are associated with larger changes in SAM (Fig. \ref{fig:sam_vs_jet_scatter}d) 
per unit poleward shift in jet position. The apparent discrepancies between trends in the SAM and
jet strength are thus resolved.  

\section{Spatial structure of historical trends} \label{sec:spatial_patterns}
The trends in SH sea level pressure and winds also have important spatial structure. The SLP
trend maps are shown over 1951 to 2004 (Fig. \ref{fig:slp_trend_maps_1951-2004}), 
because the HadSLP2r data become unreliable after 2005, as we shall see below. The HadSLP2r trend
pattern is dominated by circumpolar wide negative trends in SLP south of 50$^{\circ}$S, with 
a bullseye of strong negative trends focused over the south Pacific. To the north
HadSLP2r shows an increase in pressure near 40$^{\circ}$S, focused south of Africa. The 20CR mean trends 
shows generally very similar patterns. In the CMIP5 mean trend, there are similar circumpolar bands 
of positive trends centred
on 40$^{\circ}$S, and negative trends south of 50$^{\circ}$S. However, the CMIP5 models do not show the
focused region of large negative trends in the south Pacific, or increasing SLP south of Africa. 
In both  of these regions, the HadSLP2r trends lies outside the 2.5-97.5$^\textrm{th}$ percentile of
individual model trends, indicating  that the differences are significant 
(Fig. \ref{fig:slp_trend_maps_1951-2004}). These differences may occur because the CMIP5 models
have difficulty correctly simulating variations in the wave number 3 pattern around Antarctica 
\citep{Marshall_Bracegirdle_2014}, or because of uncertainties in the observations described below.

Winds trends are shown for 20CR and the CMIP5 mean (Fig. \ref{fig:uas_trend_maps_1951-2011}).  
20CR shows a band of large positive trends centred on the jet core near 50$^{\circ}$S, with
regions of negative trends on either side. The CMIP5 mean also shows strengthening trends,
but they are much weaker and poleward displaced relative to the 20CR trends. Thus,
the CMIP5 models show a strengthening on the poleward flank of the jet on average.  The anomaly map
shows a tripole of differences, indicating the shifted nature of the trends in the CMIP5 mean,
relative to 20CR, with the differences being significant nearly everywhere. 

In the previous sections we have shown how the CMIP5 models have trends in SLP and surface winds
that differ significantly from HadSLP2r and 20CR. These differences are evident in integrated metrics 
like the SAM index and zonal-mean jet strength, and as we have shown here are regionally focused in
the southeast Pacific. However, the southeast Pacific is one of the most data sparse regions
and significant uncertainties exist in the observations, and from the infilling methodologies 
associated with HadSLP2r \citep{Allan_and_Ansell_2006}. 

To demonstrate, the uncertainty in the 
20CR SLP and u10m trends is shown as two times the standard deviation in trends across the 56 member 
20CR ensemble (Fig. \ref{fig:20CR_uncertainty_map}). The $2 \sigma$ spread is largest in the southeast
Pacific, and it represents about 20\% of the magnitude of the mean trends. The 20CR ensemble also suffers 
from spurious trends associated with a changing observational network which are not fully quantified by 
the ensemble spread discussed above \citep{Wang_et_al_2013}. In the following sections, we address these issues 
by conducting an intercomparison of available observational and reanalysis products.

\section{Intercomparison of changes across observational products and models} \label{sec:obs-intercomp}
\subsection{SAM index computed at Marshall station locations}
One of the most reliable records of changes in the SH SLP is from the station based estimates
updated from \cite{Marshall_2003}. Data from six stations located near 40$^{\circ}$S and 
an additional six stations near 65$^{\circ}$S were averaged to give the mean SLP at those two
latitudes respectively. Here, HadSLP2r, six reanalyses, and the CMIP5 models are subsampled
at these same 12 locations in order to compare with the \cite{Marshall_2003} data.

In the timeseries of the mean pressure at 40$^{\circ}$S it can be seen that the reanalyses
and Marshall based observations have well synchronised interannual variability (Fig. \ref{fig:marshall_timeseries}a).
In all products a general long term increase in SLP at 40$^{\circ}$S is also evident. At 65$^{\circ}$S, 
the observations and all six reanalyses show a long term decline in 
SLP (Fig. \ref{fig:marshall_timeseries}b). Biases here also occur principally in R1, which starts 
with a pressure that is about 
8 hPa too high, and exhibits a large and spurious negative trend not seen in the 
observations prior to about 1990, as is well known \citep{Marshall_2003}. R2, which 
is a closely related product, has similar issues, and to a much lesser extent, 20CR. 
Since the 20CR spread is generally small after 1950 (Fig. \ref{fig:timeseries}), 
from here on we show only the 20CR ensemble mean.
It can also clearly be seen that a large and spurious change occurs after 2005 at 65$^{\circ}$S in HadSLP2r,
coincident with when that product begins to be based on R1 output, and despite efforts to 
homogenize the dataset. Hence we limit all our spatial comparisons with HadSLP2r to the period before
2005.

The CMIP5 models on average have a pressure that is systematically low by about 1 hPa at 40$^{\circ}$S
and systematically high by about 4 hPa at 65$^{\circ}$S, relative to the Marshall data 
(Fig. \ref{fig:marshall_timeseries}a, b). The SAM index shows the well known long term increase for the 
models, reanalyses and observations (Fig. \ref{fig:marshall_timeseries}c). Biases, which largely stem 
from those at 65$^{\circ}$S, are also clearly evident. To better assess the changes, SAM trends by 
season are also computed for two time periods (Fig. \ref{fig:marshall_trends}). 

Over 1958 to 2011,
trends at 40$^{\circ}$S are generally small and positive. Trends at 65$^{\circ}$S are negative and
larger, and show large biases for R1. In the SAM index, the Marshall-based trend is positive in all
seasons, except SON, when it is zero. The HadSLP2r trends also generally match the Marshall trends
well, with the largest difference occurring in SON. The 20CR ensemble mean SAM trend is slightly larger 
than the trend observed in the Marshall data in all
four seasons and the annual mean. The spread of CMIP5 trends over 1958 to 2011 includes 
the observed Marshall trend in all seasons except SON (Fig. \ref{fig:marshall_trends}). Interestingly, 
in the annual mean, the CMIP5 mean trend almost exactly matches the observed Marshall trend.

Over the shorter period from 1979 to 2009, most of the same conclusions 
hold. Trends at 40$^{\circ}$S are small and positive, while trends at 65$^{\circ}$, are negative, larger, 
and less certain. The CMIP5 range of trends includes the Marshall observations in all seasons, 
and in the annual mean the CMIP5 mean trend is again almost identical to observed over the shorter
satellite era. 

These findings suggest that there is little evidence that the CMIP5 models 
systematically  underestimate the SAM trend. This is opposite to the conclusion in 
Section \ref{sec:obs_sim_changes}, where the annual mean SAM trend (based on the zonal mean 
over all longitudes and over 1951 to 2011) in 20CR and HadSLP2r, was found to be significantly 
larger than the CMIP5 trends. Much of the reason for this is that over 1951 to 2011 the largest
SLP trends in 20CR and HadSLP2r occur in the southeast Pacific, and this region contributes
significantly to the overall SAM trend, but is also the most uncertain. In contrast, the 
Marshall based SAM index considered in this section does not have any stations located in the 
southeast Pacific, but has reliable trends due to using a fixed 
observational network \citep{Marshall_2003}. In the following section we return to examining the spatial structure of 
trends over the full Southern Ocean.

\subsection{Spatial structure of trends over the recent past}
SLP trend maps were computed for 1979 to 2004, when all reanalysis products and HadSLP2r are available
and by ending in 2004 we avoid the continuity problems in HadSLP2r identified above 
(Fig. \ref{fig:slp_trend_maps_1979-2004}). The most prominent pattern in the HadSLP2r trends
over this period is again the large negative and circumpolar trends in pressure 
south of about 50$^{\circ}$S. Similar patterns are seen in R1, R2 and 20CR, but these products
tend to overestimate the magnitude of the trends. CFSR and MERRA show the opposite, with large
positive trends, and correspondingly, these products have the largest root mean square difference
from the HadSLP2r observations. ERA-Interim also has SLP trends that are a little too positive, but 
is the best fit to the HadSLP2r observations after 20CR.  The CMIP5 models show a similar pattern 
of trends to the observations,
but generally with a weaker magnitude. Interestingly we note that the CMIP5 mean trend is more similar to
the HadSLP2r observations than any of the six reanalyses, as seen by it's small root mean square 
difference (51.65 Pa decade$^{-1}$).

Maps of u10m trends from the CCMP satellite-based wind product are compared with the 
reanalyses and CMIP5 models for the available period of 1988 to 2011 
(Fig. \ref{fig:uas_trend_maps_1988-2011}). CCMP generally shows negative trends in the zonal winds
(u10m) over the Southern Ocean during this period ($-0.13$ m~s$^{-1}$~decade$^{-1}$ averaged south of 
35$^{\circ}$S). Note that this contrasts with the surface wind-speed
trends in CCMP, which are generally positive ($+0.27$ m~s$^{-1}$~decade$^{-1}$ averaged south of 
35$^{\circ}$S) \citep{Li_et_al_2013, Wanninkhof_et_al_2013}. The CCMP u10m trend pattern 
is dominated by a large dipole like feature in
the south Pacific. All the reanalyses produce this pattern, but with varying degrees of magnitude. The
trends are generally too large in R1, R2 and 20CR. MERRA is the best fit to the CCMP observations, followed
by ERA-Interim, judged by their small root mean square difference with the CCMP trends. The CMIP5 models 
show only weak trends and do not reproduce the south Pacific dipole. 
This could reflect that fact that there is significant internal variability over the 23 year period
shown, or that the models are incapable of reproducing the correct response in the surface winds
in this region, perhaps due to their inability to capture changes in the wave-number 3 pattern
as noted above \citep{Marshall_Bracegirdle_2014}.

To help intercompare the trends discussed above, zonal mean fields of the SLP and u10m trends were
computed (Fig. \ref{fig:zonmean_trends}). In the zonal means it is clear that the CMIP5 mean reproduces
the available SLP observations very well (see red line and ``x's'' in Fig. \ref{fig:zonmean_trends}a). 
In contrast, the positive SLP trends south of 50$^{\circ}$S in
CFSR and MERRA clearly stick out as spurious. The CCMP u10m trends interestingly show no positive
trend near the peak of the jet (50-55$^{\circ}$S; Fig. \ref{fig:zonmean_trends}b).The  MERRA u10m 
trends agree fairly well with the
observations over this region, while R1, R2 and 20CR all seem to have trends that are too large.
The CCMP observations and several reanalyses also show large negative u10m trends between about 
30$^{\circ}$S and 50$^{\circ}$S. The CMIP5 model mean trend agrees well with the CCMP observations
in the region of the peak of the westerly jet near 50--60$^{\circ}$S. However the models do not
simulate the negative trends on the equatorward flank of the jet near 35$^{\circ}$S, where the CCMP 
trends fall outside the 2.5-97.5$^\textrm{th}$ percentile of the CMIP5 trends.

Because u10m winds depend on the formulation used to move winds to the reference height of 10~m
\citep{Kent_et_al_2013}, we also compare trends in surface zonal wind-stress
(Fig. \ref{fig:zonmean_trends}c). Stress fields occur at a natural level (the surface), 
but themselves depend on the drag-formulation employed. Nonetheless in general the stress fields 
convey the same picture as u10m, with R1, R2 and 20CR having
larger than observed positive trends, with MERRA and the CMIP5 mean being close to the CCMP values. 
Of note are the large negative trends evident in CFSR, consistent with a previous report 
\citep{Swart_et_al_2014}.

Clearly, the best reanalysis product depends on the time-period and variable of interest. One notable
finding is that the CMIP5 models do not seem to underestimate the jet strengthening trend relative to
the available observations, but R1, R2 and 20CR seem to overestimate the surface speed trends. In light
of this it appears that the findings of section \ref{sec:data_and_methods} that the models significantly
underestimate the jet strength trends relative to 20CR, should likely be interpreted as due to 
spuriously large trends in 20CR, not as a shortcoming in the models. This indicates that a high degree 
of caution is required in using reanalysis products to validate simulated trends. Indeed, previous studies
have also found a large spread between reanalysis products in the climatologies and 
trends of surface winds in the Southern Ocean \citep{Kent_et_al_2013, Li_et_al_2013}. In the final
section, we re-evaluate trends by season across all available products to search for robust
features of change.

\subsection{Linear trends by season over 1979 to 2009}
Here we consider trends over the 30 year period between 1979 and 2009 (Fig. \ref{fig:seas_trends_1979-2009}). 
This period has the advantage
of being well observed, due to being within the satellite era. There are also six reanalysis 
products available for comparison, and the inter-product spread allows a determination of the 
observational uncertainty. The shorter 30 year duration increases the ratio of noise 
in the trends due to internal  variability, and reduces the statistical power relative to the 
60 year period (1951-2011) used previously.  This is illustrated, for example, by the fact 
the 2.5-97.5$^\textrm{th}$ percentile spread in DJF SAM trends across the CMIP5 ensemble 
increased from 1~hPa over 1951 to 2011 to over 3~hPa over 1979 to 2009.

During DJF, all six reanalysis products, and the CMIP5 model mean show a significant positive trend
in the SAM. However, the SAM trends for the CMIP5 mean are smaller and not significant 
during the other seasons, and there is a large spread amongst the six reanalyses, 
which even differ their signs.

Similarly, the CMIP5 mean trend in jet strength is largest and statistically significant during DJF. 
All six reanalyses also show a positive trend during DJF, but the spread in magnitudes is large. 
Trends are smaller and more ambiguous during other seasons. Notably, in the annual mean, 
while the CMIP5 models show a significant positive trend on average, two reanalyses show 
negative trends, and the remaining four reanalyses have a 
factor of three spread in the magnitude of their trends.

Jet position trends show an important seasonality that is often under-appreciated. The CMIP5
mean and all six reanalyses agree that the jet shifted poleward during DJF. However, during
all the other seasons, and in the annual mean, the CMIP5 models do not show a significant trend
in position. Indeed, all six reanalyses show a near a zero trend in annual mean 
jet position during this period. The annual mean trend is near zero in the reanalyses because the 
poleward trend during DJF is balanced by opposing equatorward trends in jet position during JJA
and SON.

Jet width trends are not significant during any season for the CMIP5 mean. All six reanalyses
do show negative trends (i.e. jet narrowing) during SON, but the spread in magnitude is large,
and in the annual mean the reanalyses width trends are spread about zero.

The large spread amongst the reanalysis trends indicates the large degree of uncertainty in 
recently observed changes in the SH circulation. Similarly, the simulated changes have a 
large spread and are less certain than over the longer 60-year period. Yet, despite the overall 
uncertainty, robust changes are clear during DJF, which is expected given the combination
of ozone and GHG forcing \citep{Son_et_al_2010}. 

\section{Discussion and conclusions} \label{sec:conclusions}
Over 1951 to 2011 the DJF trends in the 20CR ensemble mean SAM index and CMIP5 multi-model 
mean are nearly identical, yet over this same period the trend in the strength of the westerly jet in
20CR is much larger than the trends seen in the CMIP5 models. Using a simple geostrophic
model we explained that trends in the SAM index and jet strength are not directly
interchangeable, because trends in jet position and width combine with changes in jet strength
to influence the SAM. For this reason, trends in the SAM should not be used as a direct
proxy for changes in any single kinematic property of the jet.

The CMIP5 models had an annual mean trend in the SAM index and jet strength
that was significantly smaller than seen in 20CR over 1951 to 2011. However, 
this is likely to partly reflect spuriously large trends in 20CR, rather than 
the CMIP5 models underestimating the true trend. Indeed, the 20CR and 
HadSLP2r SAM trends since 1951 were largely driven by 
large negative trends in SLP in the south Pacific, a data sparse region
with a large uncertainty \citep[Fig. \ref{fig:20CR_uncertainty_map};][]{Allan_and_Ansell_2006}. 

Using sea-level pressure data coincident with the 12 station locations used
by \citet {Marshall_2003}, we showed that the CMIP5 mean SLP trends 
at 40$^{\circ}$S, 65$^{\circ}$S 
and the corresponding SAM index are consistent with the direct
observations. Surprisingly, the spatial pattern of
CMIP5 model mean SLP trends was a better fit to HadSLP2r observed trends than any of six 
reanalysis products over the period 1979 to 2004. Similarly, in the zonal mean
the CMIP5 trends in jet strength since 1988 were generally consistent with the 
CCMP satellite based wind product near the core of the jet, although the models
did not reproduce the spatial pattern of changes. 20CR, R1 and R2 overestimated 
recent strengthening of the jet near its peak, relative to CCMP. 

The best performing reanalysis product depends on the variable (SLP or u10m) 
and time period of choice, but in general 20CR best reproduced observed SLP 
trends while MERRA best reproduced surface wind trends relative to observations,
and ERA-Interim performed best when combining both measures. 
However all the six reanalysis products experienced some spurious trends. 
The temporal continuity of reanalyses is inherently 
hampered by the evolving observational network which underlies these products. 
The resulting trends in Southern Hemisphere sea-level pressure and winds are 
unreliable, and as such reanalyses are likely inappropriate tools for 
validating these aspects of climate model simulations.

Many studies have used reanalysis based forcing, and particularly R1, for forcing ocean 
only models to investigate the role of Southern Ocean wind changes on ocean circulation 
\citep[e.g.][]{Screen_et_al_2009} and the carbon cycle 
\citep[e.g.][]{Le_Quere_et_al_2007, Lovenduski_et_al_2008}. The widely
used surface forcing from the Coordinated Ocean-ice Reference Experiment (CORE) phase I and II
\citep{Large_and_Yeager_2009, Griffies_et_al_2009, Danabasoglu_et_al_2014} 
is also based on R1. However, as we
have shown here, R1 has particularly large and spurious trends over the Southern 
Ocean, which might in turn bias studies using R1-derived products as surface forcing. 
Indeed, the impacts of atmospheric circulation changes on the Southern Ocean circulation
and carbon cycle are highly sensitive to the choice of surface forcing 
\citep{Swart_et_al_2014}, and the significant uncertainties associated with this forcing
require further attention.


%%%%%%%%%%%%%%%%%%%%%%%%%%%%%%%%%%%%%%%%%%%%%%%%%%%%%%%%%%%%%%%%%%%%%
% ACKNOWLEDGMENTS
%%%%%%%%%%%%%%%%%%%%%%%%%%%%%%%%%%%%%%%%%%%%%%%%%%%%%%%%%%%%%%%%%%%%%
%
\acknowledgments
We acknowledge the World Climate Research Programme's Working Group on 
Coupled Modelling, which is responsible for CMIP, and we thank the climate 
modeling groups for producing and making available their model output. For 
CMIP the U.S. Department of Energy's Program for Climate Model Diagnosis
and Intercomparison provides coordinating support and led 
development of software infrastructure in partnership with the Global 
Organization for Earth System Science Portals. Support for the Twentieth Century 
Reanalysis Project dataset is provided by the U.S. Department of Energy, Office of 
Science Innovative and Novel Computational Impact on Theory and Experiment (DOE INCITE) 
program, and Office of Biological and Environmental Research (BER), and by the National 
Oceanic and Atmospheric Administration Climate Program Office. NCEP Reanalysis data 
(R1 and R2) was provided by the NOAA/OAR/ESRL PSD, Boulder, Colorado, USA, from their 
Web site (Table \ref{t:rean_list}). CFSR and CCMP data were made available from the Research 
Data Archive at the National Center for Atmospheric Research, Computational and 
Information Systems Laboratory. GJM was supported by the UK Natural Environment Research 
Council through the British Antarctic Survey programme Polar Science for Planet Earth.

%%%%%%%%%%%%%%%%%%%%%%%%%%%%%%%%%%%%%%%%%%%%%%%%%%%%%%%%%%%%%%%%%%%%%
% APPENDIXES
%%%%%%%%%%%%%%%%%%%%%%%%%%%%%%%%%%%%%%%%%%%%%%%%%%%%%%%%%%%%%%%%%%%%%
%
% Use \appendix if there is only one appendix.
%\appendix

% Use \appendix[A], \appendix}[B], if you have multiple appendixes.
%\appendix[A]

%% Appendix title is necessary! For appendix title:
%\appendixtitle{}

%%% Appendix section numbering (note, skip \section and begin with \subsection)
% \subsection{First primary heading}

% \subsubsection{First secondary heading}

% \paragraph{First tertiary heading}

%% Important!
%\appendcaption{<appendix letter and number>}{<caption>} 
%must be used for figures and tables in appendixes, e.g.,
%
%\begin{figure}
%\noindent\includegraphics[width=19pc,angle=0]{figure01.pdf}\\
%\appendcaption{A1}{Caption here.}
%\end{figure}

%%%%%%%%%%%%%%%%%%%%%%%%%%%%%%%%%%%%%%%%%%%%%%%%%%%%%%%%%%%%%%%%%%%%%
% REFERENCES
%%%%%%%%%%%%%%%%%%%%%%%%%%%%%%%%%%%%%%%%%%%%%%%%%%%%%%%%%%%%%%%%%%%%%
% Make your BibTeX bibliography by using these commands:
 \bibliographystyle{ametsoc2014}
 \bibliography{/HOME/ncs/Documents/references}


%%%%%%%%%%%%%%%%%%%%%%%%%%%%%%%%%%%%%%%%%%%%%%%%%%%%%%%%%%%%%%%%%%%%%
% TABLES
%%%%%%%%%%%%%%%%%%%%%%%%%%%%%%%%%%%%%%%%%%%%%%%%%%%%%%%%%%%%%%%%%%%%%
%% Enter tables at the end of the document, before figures.
%%
%
%\begin{table}[t]
%\caption{This is a sample table caption and table layout.  Enter as many tables as
%  necessary at the end of your manuscript. Table from Lorenz (1963).}\label{t1}
%\begin{center}
%\begin{tabular}{ccccrrcrc}
%\hline\hline
%$N$ & $X$ & $Y$ & $Z$\\
%\hline
% 0000 & 0000 & 0010 & 0000 \\
% 0005 & 0004 & 0012 & 0000 \\
% 0010 & 0009 & 0020 & 0000 \\
% 0015 & 0016 & 0036 & 0002 \\
% 0020 & 0030 & 0066 & 0007 \\
% 0025 & 0054 & 0115 & 0024 \\
%\hline
%\end{tabular}
%\end{center}
%\end{table}

\begin{table}[t]
\caption{List of reanalyses used in this study.}\label{t:rean_list}
\begin{center}
\begin{tabular}{llll}
\hline\hline
Name & Abbreviation & Reference & Data source  \tabularnewline
\hline 
NCEP/NCAR Reanalysis 1 & R1 & \citet{Kalnay_et_al_1996} & \url{http://www.esrl.noaa.gov/psd/} \tabularnewline
NCEP/DOE Reanalysis 2 & R2 & \citet{Kanamitsu_et_al_2002} & \url{http://www.esrl.noaa.gov/psd/} \tabularnewline
Twentieth Century Reanalysis v2 & 20CR & \citet{Compo_et_al_2011} & \url{http://portal.nersc.gov/} \tabularnewline
ERA-Interim Reanalysis & ERA-Int & \citet{Dee_et_al_2011} & \url{http://apps.ecmwf.int/}  \tabularnewline
NCEP CFSR & CFSR & \citet{Saha_et_al_2010} & \url{http://rda.ucar.edu/datasets/ds093.2/}  \tabularnewline
NASA MERRA & MERRA & \citet{Rienecker_et_al_2011} & \url{http://disc.sci.gsfc.nasa.gov/} \tabularnewline
\hline
\end{tabular}
\end{center}
\end{table}

%%%%%%%%%%%%%%%%%%%%%%%%%%%%%%%%%%%%%%%%%%%%%%%%%%%%%%%%%%%%%%%%%%%%%
% FIGURES
%%%%%%%%%%%%%%%%%%%%%%%%%%%%%%%%%%%%%%%%%%%%%%%%%%%%%%%%%%%%%%%%%%%%%
%% Enter figures at the end of the document, after tables.
%%
%
\graphicspath{{../plots/}}

\begin{figure}[t]
  \noindent\includegraphics[]{timeseries.pdf}\\
  \caption{Timeseries of a) SAM index and westerly jet b) strength, c) position and d) width over 1881 to 2013.
   The shaded envelopes give the 95\% confidence interval about the mean for the 
   CMIP5 and 20CR ensembles. All data has been smoothed with a 5-year running mean.}\label{fig:timeseries}
\end{figure}

\begin{figure}[t]
  \noindent\includegraphics[]{seas_trends_1951-2011.pdf}\\
  \caption{Trends over 1951 to 2011 in a) SAM index and westerly jet b) strength, c) position and d) width
   by season in HadSLP2r, 20CR and CMIP5. The CMIP5 and 20CR ensemble mean trends are given by the 
   horizontal lines (red and green respectively), 
   the 95\% confidence interval is given by the solid vertical bars, and the 2.5-97.5$^\textrm{th}$
   percentile of trends in the individual ensemble members is given by the light vertical 
   bars. }\label{fig:seas_trends_1951-2011}
\end{figure}

\begin{figure}[t]
  \noindent\includegraphics[width=0.95\textwidth]{gaussian_jet.pdf}\\
  \caption{The a) strength, b) position and c) width of an idealized Gaussian jet and the relationship 
  between changes in the d) strength, e) position and f) width of the idealized jet and changes in the 
  corresponding SAM index.}\label{fig:gaussian_jet}
\end{figure}

\begin{figure}[t]
  \noindent\includegraphics[width=0.85\textwidth]{sam_vs_jet_scatter.pdf}\\
  \caption{The relationship between DJF trends over 1951 to 2011 in the SAM index and 
   trends in jet a) strength, c) position
   and e) width as well as climatological jet b) strength, d) position and f) width for the 30
    individual CMIP5 simulations and 20CR. Numbers in the lower right of the panels give the correlation 
    coefficient, r, and p-value of the relationship.}\label{fig:sam_vs_jet_scatter}
\end{figure}

\begin{figure}[t]
  \noindent\includegraphics[width=0.7\textwidth]{sam_v_uspd_seas_1951_2011.pdf}\\
  \caption{The relationship between trends over 1951 to 2011 in the SAM index and 
   trends in jet strength over various seasons. The $b$ values given in the lower right
   of the panels are the slope of the regression lines.}\label{fig:sam_vs_jet_seas}
\end{figure}

\begin{figure}[t]
  \noindent\includegraphics[]{slp_trend_maps_1951-2004.pdf}\\
  \caption{Trends in sea-level pressure over 1951 to 2004 for HadSLP2r, 20CR and the CMIP5 mean (left),
   and anomalies relative to the HadSLP2r trends (right). The areas where the HadSLP2r trend lies 
   outside the 2.5-97.5$^\textrm{th}$ percentile of trends in the individual CMIP5 simulations is shown 
   by stippling in the bottom right. Numbers in the right panels give the root mean square difference 
   with HadSLP2r (Pa~decade$^{-1}$).}\label{fig:slp_trend_maps_1951-2004}
\end{figure}

\begin{figure}[t]
  \noindent\includegraphics[]{uas_trend_maps_1951-2011.pdf}\\
  \caption{Trends in u10m over 1951 to 2011 (left), and anomalies relative to the 20CR trend (right). 
  The areas where the 20CR trend lies outside the 2.5-97.5$^\textrm{th}$ percentile of trends in the individual
   CMIP5 simulations is shown by stippling in the bottom right. The number in the right panel gives the 
   root mean square difference between the 20CR and CMIP5 mean trends (m~s$^{-1}$ decade$^{-1}$).}\label{fig:uas_trend_maps_1951-2011}
\end{figure}

\begin{figure}[t]
  \noindent\includegraphics[]{psl_maps_20CR_trends-uncertainty_1951-2011.pdf}\\
  \caption{The uncertainty in 20CR trends in SLP and u10m over 1951 to 2011, given
           by two times the standard deviation of trends across the 56 member 20CR ensemble.}
           \label{fig:20CR_uncertainty_map}
\end{figure}

\begin{figure}[t]
  \noindent\includegraphics[]{marshall_timeseries.pdf}\\
  \caption{Timeseries over 1962 to 2012 of pressure at a) 40$^{\circ}$S, b) 65$^{\circ}$S and 
          c) SAM index computed
          using only data from the locations of the stations used by \cite{Marshall_2003} and shown for
          the original \cite{Marshall_2003} data, HadSLP2r, six reanalyses and the CMIP5 mean plus 95\% 
          confidence interval. All data has been smoothed with a 5-year running mean.} \label{fig:marshall_timeseries}
\end{figure}

\begin{figure}[t]
  \noindent\includegraphics[width=0.95\textwidth]{marshall_trends.pdf}\\
  \caption{Trends in pressure at 40$^{\circ}$S (a, b), pressure at 65$^{\circ}$S (c, d) and the 
  SAM index (e, f), computed using only data from the locations of the stations used by 
   \cite{Marshall_2003} and shown for the periods 1958 to 2011 (a, c, e) and 1979 to 2009 
  (b, d, f). The CMIP5 mean trend is given by the horizontal red line, the 95\% confidence
   interval is given by the solid red vertical bar, and the 2.5-97.5$^\textrm{th}$ percentile of
    trends in the individual CMIP5 simulations is given by the light red vertical
     bar.}\label{fig:marshall_trends}
\end{figure}

\begin{figure}[t]
  \noindent\includegraphics[]{slp_trend_maps_1979-2004.pdf}\\
  \caption{Trends in monthly mean sea-level pressure over 1979 to 2004 for HadSLP2r, six reanalyses and
   the CMIP5 mean (left), and anomalies relative to the HadSLP2r trends (right). The areas
    where the HadSLP2r trend lies outside the 5-95$^\textrm{th}$ percentile of trends in the
     individual CMIP5 simulations is shown by stippling in the bottom right. Numbers in the 
     right panels give the root mean square difference with
      HadSLP2r (Pa~decade$^{-1}$).}\label{fig:slp_trend_maps_1979-2004}
\end{figure}

\begin{figure}[t]
  \noindent\includegraphics[]{uas_trend_maps_1988-2011.pdf}\\
  \caption{Trends in monthly mean u10m over 1988 to 2011 for CCMP satellite winds, six reanalyses and the 
  CMIP5 mean (left), and anomalies relative to the CCMP trends (right). The areas where the CCMP 
  trend lies outside the 2.5-97.5$^\textrm{th}$ percentile of trends in the individual CMIP5
   simulations is shown by stippling in the bottom right. Numbers in the right panels give
    the root mean square difference with CCMP (m~s$^{-1}$ decade$^{-1}$). }\label{fig:uas_trend_maps_1988-2011}
\end{figure}

\begin{figure}[t]
  \noindent\includegraphics[]{zonmean_trends.pdf}\\
  \caption{Trends in zonal mean a) sea-level pressure over 1979 to 2004,
  b) u10m over 1988 to 2011 and c) the u-component of the wind-stress over 1988 to 2011. The 
  solid red line shows the CMIP5 mean trend, the dark envelope shows the 95\% confidence
   interval and the light envelope shows the 2.5-97.5$^\textrm{th}$ percentile of trends in 
   the individual 
  CMIP5 simulations.}\label{fig:zonmean_trends}
\end{figure}

\begin{figure}[t]
  \noindent\includegraphics[]{seas_trends_1979-2009.pdf}\\
  \caption{Trends over 1979 to 2009 in a) SAM index, b) jet strength, c) jet position and
   d) jet width by season in six reanalyses and CMIP5. The CMIP5 mean trend is given by
    the horizontal red line, the 95\% confidence interval is given by the solid red 
    vertical bar, and the 2.5-97.5$^\textrm{th}$ percentile of trends in the individual CMIP5 
    simulations is given by the light red vertical 
  bar. Since the spread in the 20CR ensemble is very small over this period it is not shown.
  }\label{fig:seas_trends_1979-2009}
\end{figure}

\end{document}
